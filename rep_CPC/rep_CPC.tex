\documentclass[a4paper,11pt]{article}
\usepackage{stix2}
\usepackage[a4paper,scale=0.8]{geometry}
\usepackage{amsmath, ulem, bm, xcolor}

\newcommand{\cyan}[1]{\textcolor{cyan}{#1}}

\begin{document}

\noindent
Dear Editor,

Thank you for sending us the Referee reports, and we also acknowledge the Referees for their careful review, as well as the questions and comments which contributed to improvements to our revised manuscript.

We have highlighted our changes in cyan in the revised manuscript.
Attached below is our response to the Referees' questions and comments point by point.
The Referees' comments are listed in italic form, and are followed by our corresponding response.

We hope that our revised manuscript meets the criteria to be published in Computer Physics Communications.

\hfill
Yours sincerely,

\hfill
Shanshan Song, Mingyu Zhu, Hongcheng Ni and Jian Wu.

\vspace{2em}

\rule{14cm}{0.5pt}

\vspace{2em}

% ======================================================================

\section*{Referee 1}

\textit{
The authors presented a Python code to calculate the molecular structure factor in the tunneling ionization process using the weak field approximation. Tunneling ionization, indeed, is the core of the strong-field process. Although the tunneling ionization rate can be estimated by various theories, there is no simulation code available to estimate the ionization for all the common molecules. The present code bridges the gap. I did a test run of the code. It is simple and easy to be used. I recommend the publication of the work after a minor revision.
}

\vspace{1em}

We are grateful for the Referee's positive evaluation on our work.

\vspace{2em}

% ======================

\noindent\textit{
(1-1) Line 23-24 on page 2: \\
the energy of the ionizing electron -> the binding energy or orbital energy ... to avoid misleading.
}

\vspace{1em}

We thank the Referee for pointing out this misleading expression, which has been revised in the latest version of the manuscript.

\vspace{2em}

% ======================

\noindent\textit{
(1-2) Line 51 on page 4: \\
High accuracy -> the accuracy same as the SCF simulation.
Note that the present SCF is based on HF, and it is not a high-precision simulation.
}

\vspace{1em}

We acknowledge that the HF is not a high-precision method and have rephrased `high accuracy' in our manuscript.
However, we would like to emphasize that the accuracy of our result already aligns with those reported in the literature.

In addition, we have added another feature in our code by making the method for electronic structure calculations flexible.
Now, users can freely switch between HF methods or CASSCF methods to compute the molecular orbital, the latter of which would substantially improve the level of accuracy of our code.
Thereby, our code features now flexible control of accuracy.

\vspace{2em}

% ======================

\noindent\textit{
(1-3) In the program summary part, the authors should state on which machine the code is tested and how long it takes for a given example.
}

\vspace{1em}

Thanks for pointing out this issue.
We have added the following in the program summary:

{\em ``Running time: The running time depends on the size of the molecule, the basis set of the calculation, the level of precision of the electronic structure calculations, and other parameters passed to the program. The example in Fig.~2 took 1.2 seconds to finish on an AMD Ryzen 9 7950X CPU on the WSL Ubuntu 22.04 LTS.''}

\vspace{2em}

% ======================

\noindent\textit{
(1-4) Since to run the examples, matplotlib is also needed. The authors should add this in the readme.md file.
}

\vspace{1em}

Actually we already mentioned that the matplotlib is required for plotting in the readme, but not that obviously.
We have now made it clearer in the readme:

{\em ``
The program depends on the following python packages: numpy, scipy, pyscf and wigner. \\
To run the examples, matplotlib is also required for plotting.
''}

\vspace{2em}


% ======================================================================

\section*{Referee 2}

\textit{
In this work, a Python program is introduced which calculates the structure factor of molecules undergoing ionization under irradiation of intense laser light. It is based on the ionization description within the weak-field asymptotic theory (WFAT), which describes strong-field ionization as a product of functions: one function containing mainly information and parameters of the intense laser field, and a second one accounting for a molecule's structural properties. This latter factor, the molecular structure factor, is the quantity which is calculated within the herein introduced Python code. It links to another Python software package, PySCF, to calculate the molecular electronic structure and extracts from there the necessary molecular orbitals and molecular potentials.}

\textit{In general, the paper is nicely written, and the proposed software seems to be robust and accurate. I also think, it may be of use for the strong-field and attosecond community, as the calculation of the molecular structure factors is often cumbersome and hard to converge. Therefore, I can in general recommend publication of the manuscript. However, in its present form, there are some minor points which I recommend addressing.
}

\vspace{1em}

We thank the Referee for the positive assessment that our software is robust and useful.

\vspace{2em}

% ======================

\textit{
(2-1)
Concerning the references, I recommend adding work of S. Patchkovskii; L.B: Madsen and K. Doblhoff-Dier on WFAT and some of the applications towards strong-field ionization.
}

\vspace{1em}

Thanks for the suggestions.
We have added some works by the mentioned three researchers as references.

\vspace{2em}

% ======================

\textit{
(2-2)
Very often, the expression "multiplet molecules" occurs in the text. There is no such thing. I would suggest rephrasing to open-shell molecules.
}

\textit{
(2-12)
Page 11, first paragraph: why is C6H6 (benzene) labeled a "multiplet molecule"? It is closed shell singlet.
}

\vspace{1em}

We thank the Referee for pointing out the mistakes, and we have revised the manuscript accordingly.

\vspace{2em}

% ======================

\textit{
(2-3)
Page 2, Introduction, it is written: "HHG is the key to the generation of extreme ultraviolet laser pulses.:" - strictly speaking are HHG no laser pulses. I would change it to "Light pulses".
}

\textit{
(2-4)
Page 3, 1st paragraph, 1st sentence reads strangely, and I recommend rephrasing it "… the laser is manifested as photons in the frequency domain ..:"
}

\textit{
(2-5)
Page 3, after eq.1: "Eo is the energy of the ionizing electron" -> "orbital energy"
}

\textit{
(2-6)
Page 5, second line: ".. taken as the asymptotic expansion for the field strength in early stages" - unclear due to language.
}

\textit{
(2-8)
Page 7, first sentence, "The structure factor … in the integral representation is given as an integral" reads odd.
}

\vspace{1em}

Thanks for pointing out the inproper wordings. We have revised the manuscript accordingly.

\vspace{2em}

% ======================

\textit{
(2-7)
In chapter 2, the definition and arrangement of the angles (alpha, beta, gamma, z, y', x'') is not very clear. As this is definition of molecular orientation etc. is very important, I suggest putting a schematic picture there which might help understanding the definition of the coordinate system.
}

\vspace{1em}

Thanks for the helpful suggestion.
We have added a schematic diagram illustrating this rotation transformation described by the Euler angles in $z-y'-z''$ convention as Fig. 1.

\vspace{2em}

% ======================

\textit{
(2-9)
Eq 13 (and others): The authors use operator hats very inconsistently. Either use them throughout or leave them out.
}

\vspace{1em}

We thank the Referee for pointing out this issue.
We have removed all the hats related to the potential terms.

\vspace{2em}

% ======================

\textit{
(2-10)
Table 2 is not fully visible in my pdf; however, in table 2, some of the parameters are not fully clear (and are later also not properly introduced or discussed), e.g. "channel", "orient\_grid\_size" or "move\_dip\_zero", (see also my point above to the choice of the coordinate/reference system). And lmax (which sum, which eq)?
}

\vspace{1em}

We are sorry for the inconvenience. We have adjusted the column widths of Table 2 to make it fully visible.
As for the description about the parameters, we have modified our expressions in Table 2 for a clearer explanation of the parameters.

\vspace{2em}

% ======================

\textit{
(2-11)
Also, the grid levels, e.g. shortly discussed on page 10 (grid levels 1,3,5,7, ... "correspond to radial and angular sizes (40,194)...") is not clear. Only odd numbers? What are the numbers in brackets?
}

\vspace{1em}

The grid we used to evaluate the integrations is actually inherited from the \texttt{PySCF} package.
The \texttt{PySCF} uses a level number from 0 to 9 to control the fineness of the spherical grid (i.e., atom\_grid\_level in our software),
for each fineness level, it uses the corresponding radial and angular sizes of the grid.
For example, for the nitrogen molecule (N$_2$) we mentioned in Fig. 3, level 1 indicates radial and angular sizes of (40,194), that is, there are 40 points on the radial grid and 194 points on the angular grid of each N atom.

As for the grid levels present in the Figs. 3 and 4, we didn't display the results under all levels from 0 to 9 simply because the figure would otherwise be too busy to read.
Any level from 0 to 9 is accepted in the program and the users are free to choose other levels.

% ======================================================================

\vspace{5em}

Finally, in summary, we gratefully acknowledge the positive assessment from the Referees, and we hope that the revised manuscript is now ready to be published in Computer Physics Communications.


\end{document}
